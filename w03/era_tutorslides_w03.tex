%%
% This is an Overleaf template for presentations
% using the TUM Corporate Desing https://www.tum.de/cd
%
% For further details on how to use the template, take a look at our
% GitLab repository and browse through our test documents
% https://gitlab.lrz.de/latex4ei/tum-templates.
%
% The tumbeamer class is based on the beamer class.
% If you need further customization please consult the beamer class guide
% https://ctan.org/pkg/beamer.
% Additional class options are passed down to the base class.
%
% If you encounter any bugs or undesired behaviour, please raise an issue
% in our GitLab repository
% https://gitlab.lrz.de/latex4ei/tum-templates/issues
% and provide a description and minimal working example of your problem.
%%

\documentclass[
  german,            % define the document language (english, german)
  aspectratio=169,    % define the aspect ratio (169, 43)
  % handout=2on1,       % create handout with multiple slides (2on1, 4on1)
  % partpage=false,     % insert page at beginning of parts (true, false)
  % sectionpage=true,   % insert page at beginning of sections (true, false)
]{tumbeamer}


% load additional packages
\usepackage{booktabs}
\usepackage{graphicx}
\usepackage{tikz}
\usepackage{url}
\usepackage{pgfplots}
\usepackage{hyperref}
\usepackage{pmboxdraw}
\usepackage{float}
\usepackage{babel}[ngerman]
\usepackage{csquotes}[autostyle]
\usepackage[useregional]{datetime2}
\usepackage{listings}
\input{../template/riscv}

\lstset {
    frame=single,
    tabsize=4,
    breaklines=true,
    xleftmargin=5pt,
    xrightmargin=5pt,
    basicstyle=\ttfamily\footnotesize,
    %language=[RISC-V]Assembler,
}

\hypersetup { 
  colorlinks=true,
  urlcolor=blue,
  filecolor=black,
  linkcolor=black
}

% tikz  
\usetikzlibrary{fit}

% image path
\graphicspath{ {../resources/} }

% presentation metadata
\title{Übung 03: RISC-V Deep Dive}
\subtitle{Einführung in die Rechnerarchitektur}
\author{\theAuthorName}

\institute{\theGroupName\\\theSchoolName\\\theUniversityName}
\date{04. -- \DTMdisplaydate{2024}{11}{10}{-1}}

\footline{\insertauthor~|~\insertshorttitle~|~\insertshortdate}


% macro to configure the style of the presentation
\TUMbeamersetup{
  title page = TUM tower,         % style of the title page
  part page = TUM toc,            % style of part pages
  section page = TUM toc,         % style of section pages
  content page = TUM more space,  % style of normal content pages
  tower scale = 1.0,              % scaling factor of TUM tower (if used)
  headline = TUM threeliner,      % which variation of headline to use
  footline = TUM default,         % which variation of footline to use
  % configure on which pages headlines and footlines should be printed
  headline on = {title page},
  footline on = {every page, title page=false},
}


% available frame styles for title page, part page, and section page:
% TUM default, TUM tower, TUM centered,
% TUM blue default, TUM blue tower, TUM blue centered,
% TUM shaded default, TUM shaded tower, TUM shaded centered,
% TUM flags
%
% additional frame styles for part page and section page:
% TUM toc
%
% available frame styles for content pages:
% TUM default, TUM more space
%
% available headline options:
% TUM empty, TUM oneliner, TUM twoliner, TUM threeliner, TUM logothreeliner
%
% available footline options:
% TUM empty, TUM default, TUM infoline


\begin{document}

\maketitle

\begin{frame}[c]{Mitschriften \& Infos}{}
  \begin{minipage}[t]{\textwidth}
    \begin{columns}[c]
      \begin{column}{0.8\textwidth}
        Montags: \href{\zulipMo}{\zulipMo}
      \end{column}
      \begin{column}{0.2\textwidth}
        \includegraphics[width=0.8\linewidth]{\zulipMoQrFilename}
      \end{column}
    \end{columns}
  \end{minipage}
  \rule{\textwidth}{0.4pt}
  \begin{minipage}[t]{\textwidth}
    \begin{columns}[c]
      \begin{column}{0.8\textwidth}
        Donnerstags: \href{\zulipDo}{\zulipDo}
      \end{column}
      \begin{column}{0.2\textwidth}
        \includegraphics[width=0.8\linewidth]{\zulipDoQrFilename}
      \end{column}
    \end{columns}
  \end{minipage}
  \ifdefined\myWebsite
  \rule{\textwidth}{0.4pt}
  \centering
  Website: \href{\myWebsite}{\myWebsite}
  \fi
\end{frame}

\begin{frame}[c]{}{}
  \begin{center}
    \LARGE  Keine Garantie für die Richtigkeit der Tutorfolien.

    \Large Bei Unklarheiten/Unstimmigkeiten haben VL/ZÜ-Folien recht!
  \end{center}
\end{frame}

\begin{frame}[c]{Inhaltsübersicht}{}
  \begin{columns}[c]
    \begin{column}{1\textwidth}
      \begin{itemize}
        \item Quiz
        \item Kurze Wiederholung
        \item Tutorblatt
        \begin{itemize}
          \item Arrays und deren Adressierung
          \item Zeichenketten/Strings
          \item Taschenrechner-Tester (Präsenzaufgabe 01)
        \end{itemize}
      \end{itemize}
    \end{column}
  \end{columns}
\end{frame}

\begin{frame}[c]{}{}
  \begin{center}
    \LARGE  Wiederholung
  \end{center}
\end{frame}

\begin{frame}[c]{Register \& Calling Convention}{}
  \begin{columns}[c]
    \begin{column}{0.5\textwidth}
      \begin{itemize}
        \item Argumente: a0 bis a5
        \item Rückgabewert: In a0 - a1 erwartet
        \item Temporäre Register: t0 - t6 können einfach überschrieben werden
        \item Saved Registers: s0 - s11 können genutzt werden, müssen vor return aber wiederhergestellt werden
      \end{itemize}
    \end{column}
    \begin{column}{0.5\textwidth}
      \includegraphics[width=\linewidth]{riscv_registers.png}
    \end{column}
  \end{columns}
\end{frame}

\begin{frame}[c]{Wichtige Instruktionen}{}
  \centering
  \includegraphics[width=\linewidth]{w03_integerinstructions.png}
\end{frame}

\begin{frame}[c]{If-else in RISC-V}{}
  \centering
  \includegraphics[width=0.75\linewidth]{w03_ifelseassembly.png}
\end{frame}

\begin{frame}[c]{While in RISC-V}{}
  \centering
  \includegraphics[width=0.8\linewidth]{w03_whileassembly.png}
\end{frame}

\begin{frame}[c]{Speicherorganisation}{}
  \begin{columns}[c]
    \begin{column}{0.5\textwidth}
      \begin{itemize}
          \item Speicher ist üblicherweise byte-adressierbar -- bei uns auch
          \item Speicher unterteilt in Zellen; Zellengröße entspricht Verarbeitungsgröße
      \end{itemize}
    \end{column}
    \begin{column}{0.5\textwidth}
      \includegraphics[width=0.75\linewidth]{speicher.png}
    \end{column}
  \end{columns}
\end{frame}

\begin{frame}[c]{Strings sind Arrays}{}
  \begin{columns}[c]
    \begin{column}{0.5\textwidth}
      \begin{itemize}
        \item Strings sind nur Arrays aus Buchstaben
        \item Im Computer sind die Buchstaben durch Zahlen repräsentiert
        \item Zeichen sind ASCII enkodiert (0-127)
        \item 2 verschiedene Arten von Strings
        \begin{itemize}
          \item C-Strings: Aneinanderreihung von Zeichen dann NULL-Byte
          \item Pascal-Strings: Erst 1 Byte für die Länge des Strings, dann Zeichen
        \end{itemize}
      \end{itemize}
    \end{column}
    \begin{column}{0.5\textwidth}
      \includegraphics[width=\linewidth]{ascii.png}
    \end{column}
  \end{columns}
\end{frame}

\begin{frame}[c]{Structs}{}
  \begin{itemize}
    \item Arrays für Daten gleiches Typs <=> \texttt{struct}s fur Daten unterschiedlichen Typs 
    \item Assembler kennt keine Typen -- nur Zahlen in verschiedenen Größen
    \item \texttt{la} = load address. Pseudoinstruktion, die Adresse in Register ladt
  \end{itemize}
  \vspace{0.4cm}
  \includegraphics[width=\linewidth]{strutcs_comparison.png}
\end{frame} 

\begin{frame}[c]{Sections}{}
  \begin{columns}[c]
		\begin{column}{.65\textwidth}
			wichtige sog. Sections:
			\begin{enumerate}
				\item \texttt{.text}\\
				beinhaltet ausführbaren Code
				\item \texttt{.data}\\
				beinhaltet globale, schreibbare, initialisierte Daten
				\item \texttt{.rodata}\\
				beinhaltet globale, nicht-schreibbare, initialisierte Daten
				\item \texttt{.bss} (= block starting symbol)\\
				beinhaltet globale, schreibbare, nicht-initialisierte Daten\\
				effektiv nur ein Marker für \enquote{hier kommen \(n\) Bytes}
			\end{enumerate}
		\end{column}
		\begin{column}{.3\textwidth}
			\vspace*{-2em}
			\centering
      \includegraphics[width=0.8\linewidth]{sections.png}
		\end{column}
	\end{columns}
\end{frame}


\begin{frame}[fragile]{\texttt{.data}-Section in RISC-V}{}
  \includegraphics[width=1.0\linewidth]{data-section.png}
\end{frame}

\begin{frame}[c]{}{}
  \begin{center}
    \LARGE Fragen?
  \end{center}
  \vspace{0.5cm}
  \begin{center}
    \LARGE Bis zum nächsten Mal ;) \\
  \end{center}
  \vspace{1.0cm}
  \begin{center}
    \small Folien inspiriert von Markus Gschoßmann und Clemens Schwarzmann
  \end{center}
\end{frame}

\end{document}